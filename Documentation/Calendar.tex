%RESERVED CHARACTERS: \ # $ % & _ ^ { } ~
%
%NEWCOMMMAND FORMAT: A simple substitute.
% \newcommand{<\NewCommandName>}{text}
%EX: \newcommand{\stamp}{\hspace{.5in}\textbf{Findings:}%
%                        \hspace{.5in}}
%
%NEWENVIRONMENT FORMAT: Involving a defined Environment Name.
% \newenvironment{<NewEnvironmentName>}%
%{\begin{<OldEnvironmentName>}<new stuff>{/end{OldEnvironmentName>}}
%EX: \newenvironment{smallcapit}%
%               {\begin{itemize} \scshape}{\end{itemize}}
%
%FORMAT OF A MACRO WITH PLACEHOLDERS:
% \newcommand{<\CommandName>}[#]{A single arguement: text and #.}
%EX: \newcommand{\phonebk}[3]{NAME: #1$|$ TEL: #2$|$ FAX: #3\\}

\documentclass[letterpaper]{article}

\usepackage{fancyhdr}
\usepackage{graphicx}
\usepackage{calc}
\usepackage[usenames,dvipsnames]{color}
\usepackage{multicol}

%PAGE LAYOUT SETTINGS (based on 8.5'' by 11'' paper letterpaper)
%Set page so pages have a 1'' border all around
%header is from 1'' to 1.25'' from top and 0.25''above text
\setlength{\topmargin}{0in}
\setlength{\headheight}{0.25in}
\setlength{\headsep}{0.25in}
%body text is 9''high and 6.5'' wide
\setlength{\hoffset}{-0.25in}
\setlength{\voffset}{-0.5in}
\setlength{\oddsidemargin}{0in}
\setlength{\evensidemargin}{0in}
\setlength{\textheight}{9.0in}
\setlength{\textwidth}{6.5in}
%footer is from 1'' to 1.25'' from bottom and 0.25'' below text
\setlength{\footskip}{0.5in}

%PUNCTION ABBREVIATIONS
\newcommand{\BS}{$\backslash$}   %backslash
\newcommand{\LB}{$\{$}           %left brace
\newcommand{\RB}{$\}$}           %right brace
\newcommand{\SP}{\ }             %space
\newcommand{\LSQ}{`}             %left single quote
\newcommand{\LDQ}{``}            %left double quote
\newcommand{\RSQ}{'}             %right single quote
\newcommand{\RDQ}{''}            %right double quote

%LIST COMMANDS
\newcommand{\bi}{\begin{itemize}}
\newcommand{\ei}{\end{itemize}}
\newcommand{\be}{\begin{enumerate}}
\newcommand{\ee}{\end{enumerate}}
\newcommand{\bd}{\begin{description}}
\newcommand{\ed}{\end{description}}

%PRINT COMMANDS
\newcommand{\prbf}[1]{\textbf{#1}}      %print in bold
\newcommand{\prit}[1]{\textit{#1}}      %print in italic
\newcommand{\prmd}[1]{\textmd{#1}}      %print in medium
\newcommand{\prno}[1]{\textnormal{#1}}  %print in default font
\newcommand{\prrm}[1]{\textrm{#1}}      %print in roman family
\newcommand{\prsc}[1]{\textsc{#1}}      %print in small cap
\newcommand{\prsf}[1]{\textsf{#1}}      %print in sans serif
\newcommand{\prsl}[1]{\textsl{#1}}      %print in slant
\newcommand{\prtt}[1]{\texttt{#1}}      %print in typewriter
\newcommand{\prup}[1]{\textup{#1}}      %print in straight up

%VERBATIM AND IGNORE
\newcommand{\bv}{\begin{verbatim}}
\newcommand{\V}{\verb} %Ex:  \V=-d{#@~}= Expr must fit on a line

% Write your own instructions, aliases macros and abbreviations here
% Time display
% -------------
% \time is minutes since midnight

\newcounter{hours}  \newcounter{minutes}
\newcommand{\printtime}{%
\setcounter{hours}{\time/60}%
\setcounter{minutes}{\time-\value{hours}*60}%
 \thehours :\theminutes}
% Need to figure out how to have print be xx:xx format

%<!-- ----- ----- ----- ----- ----- ----- ----- ----- ----- ----- ----- ----->
\title{Calendar Program}
\author{Edward Birdsall}
\date{\today}

%FOR PRINTOUTS WITH EMPTY LINES BETWEEN PARAGRAPHS AND NO INDENT AT START
\flushbottom
\parindent=0pc
\setlength{\parskip}{1pc}

\begin{document} %End of preamble. Begin writing.
\maketitle %Produces the title

\section{Introduction}

This program set are programs to display and printout a calendar of events for the user.  It should be able to display a number of different calendars both real and fictional.  
The python verion is intended to have a number of different UIs which the C version is primarily a command line interface and the Fortran version for use in programs.  I've 
written bits and pieces in various computing languages over the decades and have lost some versions so this is an attempt to get it all together, written and working on the 
linux platform.  Another hope is to do it well enough in python to make it portable to other platforms.  Porting to other operating systems is low priority for me but may be
done as an exercise.

Different calendars desired: \vspace{-.2in}
     \bi
     \item Gregorian  - ideal and convert to Julian for dates before 10/15/1582  (10/5/1582 Julian == 10/15/1582 Gregorian begin)  \vspace{-.1in}
     \item Julian \vspace{-.1in}
     \item Shire \vspace{-.1in}
     \item Oriental \vspace{-.1in}
     \item Jewish \vspace{-.1in}
     \item Scientific  - 13 months of 28 days, midDay is between the 14th and 15th of the seventh month as well as LeapDay both of which are not part of the week  \vspace{-.1in}
     \item perpetual - similar to Scientific  \vspace{-.1in}
     \item Mars  MDay=24hrs,39min 35 seconds  Myear is 668.82 MDays = 687 EDays \vspace{-.1in}
     \item lunar \vspace{-.1in}
     \item lunar \vspace{-.1in}
     \item Various other proposed calendar ideas
     \ei
     
User Interfaces for the program set; Graphical, Web Browser and Command Line.  Depending on the user interface not all features may be available.  Below is a listing of various
program features and subfeatures; \vspace{-.25in}

\bi     
\item printout a calendar \vspace{-.1in}
     \bi
     \item Day's events \vspace{-.1in}
     \item by week \vspace{-.1in}
     \item by month \vspace{-.1in}
     \item by year \vspace{-.1in}
     \item blank calendar (month, year) \vspace{-.1in}
     \item for calendars week long or greater give an option as to which is the starting day of the week \vspace{-.1in}
     \ei
\item convert between calendars \vspace{-.1in}
\item account for time zones based on location \vspace{-.1in}
\item schedule an event \vspace{-.1in}
\item be able to invite other people by e-mail-ics-gcal \vspace{-.1in}
\item be able to import events from iCal and other calendars \vspace{-.1in}
\item printout or email day's events to user using preferred calendar structure or random calendar \vspace{-.1in}
\item support multiple calendars for each user
\ei

\section{Calendar Information}
\subsection{Gregorian}
	365 days per year except leap year has 366.  Has 12 months with 28, 29, 30 or 31 days.  Months are;
	January(31 Days), February(28,29 days), March(31 days), April(30 days), May(31 days), June(30 days), July(31 days), 
	August(31 days), September(30 days), October(31 days), November(30 days), and December(31 days).
		Converted from Julian calendar on 10/15/1582 Julian becoming 10/15/1582 Gregorian in some locations.
\subsection{Julian}
		365 days per year with 12 months with 28, 29, 30 or 31 days.  Months are;
	January(31 Days), February(28 days), March(31 days), April(30 days), May(31 days), June(30 days), July(31 days), 
	August(31 days), September(30 days), October(31 days), November(30 days), and December(31 days).
		Converted from Julian calendar on 10/15/1582 Julian becoming 10/15/1582 Gregorian in some locations.
 \subsection{Shire}
The Shire calendar found in J.R.Tolkein Ring series has twelve months of thirty
days with four days not in a month and one or two not in a month or week.  Each
year starts on 2Yule which is Sterrendei (equivalent to Saturday).  The year
pattern is 2Yule, 1-30 Afteryule, 1-30 Solmath, 1-30 Rethe, 1-30 Astron, 1-30
Thrimidge, 1-30 Forelithe, 1Lithe, [Midyear's Day, (Overlithe)], 2Lithe, 1-30
Afterlithe, 1-30 Wedmath, 1-30 Halimath, 1-30 Winterfilth, 1-30 Blotmath, 
1-30 Foreyule, 1Yule.  Overlithe added for leap years and Midyear's Day are not
part of any week.

Days of the week are; Sterrendei, Sunnendei, Momendei, Trewesdei,Hevenesdei,
Meresdei, and Highdei. 
\subsection{Oriental}
Chinese Zodiac animals - monkey, rooster, dog, pig, rat, ox, tiger, rabbit, dragon, snake, horse, sheep 
\subsection{Jewish}

\bi
	\item A month is calculated as 29 days, 12 hours, and 793 "parts"  \vspace{-.15in}
	\item Leap years occur in years 3, 6, 8, 11, 14, 17 and 19 of a 19-year cycle \vspace{-.15in}
	\item Adjustments (dechiyot) prevent round off the date calculated \vspace{-.15in}
	\item Dechiyot prevent oddities in the length of the year  \vspace{-.15in}
	\item  Dechiyot prevent holidays from falling on the wrong day of the week  \vspace{-.15in}
	\item  Some months have variable lengths \vspace{-.15in}
	\item There are 14 possible formats of year, identified by codes \vspace{-.15in}
	\item  The calendar is not perfect, but it is very accurate  \vspace{-.15in}
\ei
\subsection{Scientific / International fixed}
13 months of 28 days which starts on Sunday, \\
midDay is between the 14th and 15th of the seventh month as well as LeapDay both of which are not part of the week
\subsection{perpetual?}
This calendar was proposed a while ago and I don't remember all the details.
It has thirteen months of twenty eight days with MidDay between the fouteenth and
fifteenth day of the seventh month.  MidDay and LeapDay which is added on leap
years are not part of the week similar to the Shire Calendar.
\subsection{Mars}
MDay=24hrs,39min 35 seconds  Myear is 668.82 MDays = 687 EDays
\subsection{Earth Lunar}
\subsection{Various Other proposed calendars}

\section{Calendar Structures}
Date - Year - Day of Year
Time - Zulu time zone (Greenwich Mean time) - Seconds from midnight

Calendar Event Structure
\bi
 	\item date and time start \vspace{-.15in}
	\item date and time end \vspace{-.15in}
	\item duration \vspace{-.15in}
	\item location - ?URL to location map \vspace{-.15in}
	\item originator \vspace{-.15in}
	\item invitees \vspace{-.15in}
	\item invitees status \vspace{-.15in}
	\item description/agenda \vspace{-.15in}
	\item date and time created \vspace{-.15in}
	\item frequency - interval and end criteria??
\ei
\section{Calendar Routines/functions/...}
\subsection{Common}


\subsection{Gregorian}
Gregorian Calendar
12 months of varying days and various starting days

month names = { January, February, March, April, May, June, July, August, September,
				October, November, December }
days names = {Sunday, Monday, Tuesday, Wednesday, Thursday, Friday, Saturday}
   Gregorian                                                                     \\ 
       Year - Month - Day of Month - Time - time zone - Day of Week            \\ 
                This calendar was started in year 1582 A.D. with a date shift    \\ 
                   from the existing calendar so need to put that in the       \\ 
                     algorithm.  10/5/1582==10/15/1582                           \\ 
                                                                              \\ 
         Year - Christian calendar wih A.D. and B.C.                             \\ 
      Month - 12 months of varying days, second month changes number of       \\ 
                 days if leap year                                               \\ 
            January-31, February-28?29, March-31, April-30, May-31, June-30    \\ 
              July-31, August-31, September-30, October-31, November-30,         \\ 
            December-31                                                        \\ 
         Day of Month - see above for days in each month                         \\ 
       Time - Earth 24 hour time                                               \\ 
         Time Zone - ranges from -12 to +12 with 0 being Zulu                    \\ 
       Day of Week - 7 days per week,  each year starts on a different day     \\ 
              Sunday, Monday, Tuesday, Wednesday, Thursday, Friday, Saturday     \\ 

enum Gdow {Sunday, Monday, Tuesday, Wednesday, Thursday, Friday, Saturday};
enum GMon {January, February, March, April, May, June, July, August, September, October, November, December}
enum Gry {31,28,31,30,31,30,31,31,30,31,30,31}
enum Gly {31,29,31,30,31,30,31,31,30,31,30,31}
Years divisible by 4 a leap year except divisible by 100 unless divisible by 400
logical isaLeapYear(int year)
{
	If (year mod 4) <> 0 then
		isaLeapYear = FALSE;
	else 
		isaLeapYear = TRUE;
	If (((year mod 100) == 0) and ((year mod 400) <> 0)) then
		isaLeapYear = FALSE;
	return isaLeapYear;
}
\subsection{ Julian}

       Year - Day - Time
 
     Year - could be Christian or some other - pick is Christian
     Day - First day of the year is 1 and then sequentially numbered till end of year
     Time - Earth 24 hour day at Zulu (Greenwich Mean Time) 
           24 hours/day, 60 minutes/hour, 60 seconds/minute 

\subsection{Shire}
12 months of 30 days
year starts on (Saturday) Sterrendei  with 2 Yule

month names = { Afteryule, Solmath, Rethe, Astron, Thrimidge, Forelithe,
				Afterlithe, Wedmath, Halimath, Winterfilth, Blotmath, Foreyule}
Days names = {Sterrendei, Sunnendei, Momendei, Trewesdei, Hevenesdei, Meresdei, Highdei }

Year =>2Yule, Afteryule, Solmath, Rethe, Astron, Thrimidge, Forelithe, 1Lithe,
       Midyear's Day, (Overlithe),		[Not days of any week, overlithe for leap year]
       2Lithe, Afterlithe, Wedmath, Halimath, Winterfilth, Blotmath, Foreyule, 1Yule

   Shire                                                                         \\ 
       Year - Month - Day of Month - Time - time zone - Day of Week            \\ 
                                                                                 \\ 
      Year - ?    \\ 
         Month - 12 months of 30 days - 5 days not part of any month or week     \\ 
           Afteryule, Solmath, Rethe, Astron, Thrimidge, Forelithe,           \\ 
              Afterlithe, Wedmath, Halimath, Winterfilth, Blotmath, Foreyule     \\ 
       Day of month - 1 to 30 with a few days not part of any month            \\ 
         Time - Earth 24 hour time                                               \\ 
       Time Zone - ranges from -12 to +12 with 0 being Zulu                    \\ 
         Day of Week - 7 days per week, each year starts on Sterrendei           \\ 
            Sterrendei, Sunnendei, Momendei, Trewesdei, Hevenesdei,            \\ 
              Meresdei, Highdei                                                  \\ 
      Year's progression                                                      \\ 
              2Yule, Afteryule, Solmath, Rethe, Astron, Thrimidge, Forelithe,    \\ 
           1Lithe,                                                            \\ 
              Midyear's Day, (OverLithe), - these two are not part of the week   \\ 
            2Lithe, Afterlithe, Wedmath, Halimath, Winterfilth, Blotmath,      \\ 
              Foreyule, 1Yule                                                    \\ 
\begin{verbatim}
int Shire2internal(gdates itime, edaet ftime)int gregorian2internal(gdates itime, edaet ftime)
{
	int	errNum = 0;
	int maxdays;
	struct etyme theTime;
	
	isleap(itime.year) ? maxdays = 366 : maxdays = 365;
	if (itime.zone != 0)
	{
		errNum = local2zulu(itime.etyme, itime.zone, theTyme);
		if (theTyme.hour > 23) 
		{
			itime.day += 1;
			theTyme.hour -= 24;
		}
		if (theTyme.hour < 0 )
		{
			itime.day -= 1;
			theTyme.hour += 24;
		}
	}
	errNum = etime2seconds(theTyme, ftime.seconds);
	ftime.year = itime.year;
	switch (itime.month)
	{
		case  0: \\ Days outside of months
				if (itime.day == 1) ftime.days = 1;
		case  1: \\ Yule
				ftime.days = itime.day + 1;
				if (itime.day < 0 )
				{
					isleap((itime.year - 1)) ? ftime.days = 366 : ftime.days = 365;
					ftime.year -= 1;
				}
				break;
		case  2: \\ Solmath
				ftime.days = itime.day + 31;
				break;
		case  3: \\ Rethe
				ftime.days = itime.day + 61;
				break;
		case  4: \\ Astron
				ftime.days = itime.day + 91;
				break;
		case  5: \\ May - 31
				ftime.days = itime.day + 121;
				break;
		case  6: \\ June - 30
				ftime.days = itime.day + 151;
				break;
		case  7: \\ July - 31
				isleap(itime.year) ? ftime.days = 181 + itime.day : ftime.days = 182 + itime.day;
				break;
		case  8: \\ August - 31
				isleap(itime.year) ? ftime.days = 212 + itime.day : ftime.days = 213 + itime.day;
				break;
		case  9: \\ September - 30
				isleap(itime.year) ? ftime.days = 243 + itime.day : ftime.days = 244 + itime.day;
				break;
		case 10: \\ October - 31
				isleap(itime.year) ? ftime.days = 273 + itime.day : ftime.days = 274 + itime.day;
				break;
		case 11: \\ November - 30
				isleap(itime.year) ? ftime.days = 304 + itime.day : ftime.days = 305 + itime.day;
				break;
		case 12: \\ December - 31
				isleap(itime.year) ? ftime.days = 334 + itime.day : ftime.days = 335 + itime.day;
				if (ftime.days > maxdays)
				{
					ftime.year += 1;
					ftime.days = 1;
				}
				break;
	}
	return (errNum);
}

\end{verbatim}
\subsection{Oriental}
\begin{verbatim}
year = int(input("Enter a year: "))
zodiacYear = year % 12 
if zodiacYear == 0:
    print("monkey")
elif zodiacYear == 1:
    print("rooster")
elif zodiacYear == 2:
    print("dog")
elif zodiacYear == 3:
    print("pig")
elif zodiacYear == 4: 
    print("rat")
elif zodiacYear == 5: 
    print("ox")
elif zodiacYear == 6:
    print("tiger")
elif zodiacYear == 7:
    print("rabbit")
elif zodiacYear == 8:
    print("dragon")
elif zodiacYear == 9:
    print("snake")
elif zodiacYear == 10:
    print("horse")
else: 
    print("sheep")
\end{verbatim}
\subsection{ Jewish}
\bi
	\item A month is calculated as 29 days, 12 hours, and 793 "parts"
	\item Leap years occur in years 3, 6, 8, 11, 14, 17 and 19 of a 19-year cycle
	\item Adjustments (dechiyot) prevent round off the date calculated
	\item Dechiyot prevent oddities in the length of the year
	\item  Dechiyot prevent holidays from falling on the wrong day of the week
	\item  Some months have variable lengths
	\item There are 14 possible formats of year, identified by codes
	\item  The calendar is not perfect, but it is very accurate
\ei

\subsection{ Scientific/International Fixed}
   Scientific	/  International Fixed         \\ 
       Year - Month - Day of Month - Time - time zone - Day of Week            \\ 
                                                                                 \\ 
       Year - ?							         \\ 
         Month - 13 months						         \\ 
       Day of month - 1 through 28 with MidDay and LeapDay not part of month   \\  
         Time - Earth 24 hour time                                               \\ 
       Time Zone - ranges from -12 to +12 with 0 being Zulu                    \\ 
         Day of Week - 7 days per week,  each year starts on Sunday              \\ 
            Sunday, Monday, Tuesday, Wednesday, Thursday, Friday, Saturday     \\ 
            January, February,March, April,May,June,Sol,July,August,September,October,November,December
              YearDay and LeapDay not part of any week                            \\ 
            YearDay is at the end of December               \\ 
              LeapDay is between June and Sol on leap years                               \\ 

\subsection{perpetual}
   perpetual
       Year - Month - Day of Month - Time - time zone - Day of Week            \\ 
                                                                                 \\ 
       Year - ?
         Month - 12 months 8-30 4-31 
       Day of month - 1 through 28 with MidDay and LeapDay not part of month   \\  
         Time - Earth 24 hour time                                               \\ 
       Time Zone - ranges from -12 to +12 with 0 being Zulu                    \\ 
         Day of Week - 7 days per week,  each year starts on Sunday              \\ 
            Sunday, Monday, Tuesday, Wednesday, Thursday, Friday, Saturday     \\ 
              MidDay and LeapDay not part of any week                            \\ 
            WorldDay is last day of year except every 4th year                 \\ 
              LeapDay is with MidDay on leap years                               \\ 
\subsection{ Mars }
   Mars1    \\ 
  MDay = 24 hours, 39 minutes, 35 seconds of ETime    \\ 
    MYear =  668.62... MDays    = 687 EDays     \\ 

\subsection{ lunar}
\subsection{Various other proposed calendar ideas}

\end{document}